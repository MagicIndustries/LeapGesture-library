\documentclass[11pt,a4paper,polish,thesis]{dcsbook}

\usepackage[utf8]{inputenc}
\usepackage{babel}
\setcounter{secnumdepth}{4}
\setcounter{tocdepth}{3}

\begin{document}

\author{Katarzyna Zjawin, Michał Nowicki, Olgierd Pilarczyk, Jakub Wąsikowski}
\title{Biblioteka do rozpoznawania gestów dla kontrolera Leap Motion}
\supervisor{dr~inż.~Wojciech Jaśkowski}
\date{Poznań, 2013}
\maketitle
\frontmatter
\tableofcontents{}
\mainmatter

\chapter{Wstęp}

Wprowadzenie do tematu...

\section*{Cel i zakres pracy}

Celem niniejszej pracy jest...

\chapter{Podstawy teoretyczne}

Więcej informacji można znaleźć w książce \cite{sop}.

\chapter{Projekt systemu}

\chapter{Implementacja systemu}

\chapter{Testy efektywnościowe}

\chapter{Zakończenie}

\appendix

\chapter{Przewodnik użytkownika}

\backmatter

\begin{thebibliography}{1}
\bibitem{sop}A.~Tanenbaum. \emph{Operating Systems Design and Implementation}.
Prentice Hall, 2006.
\end{thebibliography}

\end{document}
