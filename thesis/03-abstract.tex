
\chapter{Abstract}
Nowadays, when computers are ubiquitous, people need to develop more natural and intuitive interface to use computer. People would like to use gestures of everyday life and translate them into a virtual world. This paper presents library for gesture recognition dedicated to Leap Motion Controller called LMGesture.  Leap Motion is a new device, which tracks fingers and other objects up to 1/100th of a millimeter. The LMGesture library can be used for various kinds of gestures. This can be done, because of the use of different detection methods for different types of gestures. For static gestures recognition has been used support vector machine (SVM) and for dynamic gestures -- hidden Markov model (HMM). In library can be found additional modules: recorder -- for recording gestures in a format supported by the library, visualizer -- for reviewing recorded gestures, finger differentiation module -- for differentiating fingers in a performed gesture.  In this paper has been also presented classification of gestures and its modification in the context of Leap Motion Controller. Additionally this work includes descriptions of helper methods used for the gesture recognition, such as pre-processing of data obtained from the device, which gets rid of existing noise or method for fingers differentiating, whereby the obtained results are more accurate.

{\color{red} [dodac opis testow]}



Since computers became small enough to fit on a desk and fast enough to provide a high resolution graphical interface instead of a text--based one, a need for a intuitive and natural human--computer interface has appeared. This trfesis presents a library which enables a computer user to control it using one of the most natural forms of human expression -- gestures. It is designed to use Leap Motion -- an innovative 3D motion capturing device, designed especially for hands tracking. With precision levels claimed by manufacturer up to 1/100th of a milimeter, small dimensions and relatively low price, Leap Motion has a potential to grab a significant part of human--computer interfaces market. This work presents a library created to make task of recognizing gestures as simple as possible for application developers. The library consists of two methods of gesture recognition: Support Vector Machine (SVM), for ''static'' gestures (defined by layout of fingers seen by controller) and hidden Markov models (HMM) for ''dynamic'' gestures (various shapes drawn in air). It provides a clear and convenient high--level interface for C++ language, a gesture recorder -- to store gestures for further processing, data preprocessor -- to reduce the impact of noise or measurement inaccuracies, visualiser -- a 3D player for stored ngestures and learning module -- to train SVM and HMM clasifiers using custom set of gestures.