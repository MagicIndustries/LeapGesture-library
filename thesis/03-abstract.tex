\chapter{Streszczenie}
Od czasu wynalezienia komputerów istnieje konieczność opracowania bardziej intuicyjnych metod interakcji człowieka z komputerem.
Wykorzystywanie urządzeń takich jak klawiatury oraz myszki nie jest tak naturalne jak gesty, które pozwalają wyrażać ludzką ekspresję.
Niniejsza praca bada nowe możliwości zastosowania gestów do interakcji człowieka z komputerem, które pojawiły się wraz z urządzeniem Leap Motion.
W pracy zaproponowano również nową klasyfikację gestów dostosowaną stricte do rozpoznawania gestów, która tworzy podział na gesty akcji oraz gesty parametryzowane.
Leap Motion to innowacyjne urządzenie, które zostało zaprojektowane do przechwytywania danych dotyczących rąk oraz palców w trójwymiarze z dokładnością do $0,01mm$.
Autorzy pracy zbadali możliwości tego sensora pod kątem rozpoznawania gestów statycznych i dynamicznych.
Statyczne gesty, które rozumiane są jako układ dłoni, są rozpoznawane za pomocą Maszyny wektorów nośnych (SVM) z wykorzystaniem wstępnego i końcowego przetwarzania danych.
Zaproponowane podejście pozwoliło na rozpoznawanie pięciu gestów z dokładnością do $99\%$ oraz dziesięciu gestów z dokładnością do $85\%$.
Gesty dynamiczne, które opierają się na ruchu ręki oraz palców, są rozpoznawane za pomocą Ukrytych modeli Markowa (HMM).
Przyjęte podejście pozwoliło na osiągnięcie poziomu rozpoznawania pięciu gestów rzędu $80\%$.
Praca zawiera również badania związane z modułem rozróżniania palców, który osiąga dokładność rozpoznawania do $93\%$. Rozpoznawanie palców jest również oparte o SVM.
Głównym rezultatem niniejszej pracy jest przeznaczona dla programistów biblioteka LeapGesture typu open-source dedykowana dla kontrolera Leap Motion. Biblioteka, oprócz rozpoznawania standardowych gestów akcji, wspiera również gesty parametryzowane. W bibliotece zostały zaimplementowane wyżej wymienione podejścia przy użyciu języka wysokiego poziomu C++, dzięki czemu użycie LeapGesture jest proste dla każdego typu aplikacji.

\chapter{Abstract}
Since the invention of computers there exists a need to develop more intuitive human-computer interfaces.
There has been keyboards, mouses, but those solutions are not as natural as gestures, which are essential part of a human expression.
This thesis studies the new possibilities to gesture interfaces that emerged with a Leap Motion sensor.
The work also proposes a new gesture classification into action and parameterized gestures, which is defined strictly in the context of gesture recognition.
The Leap Motion is an innovative, 3D motion capturing device designed especially for hands and fingers tracking with precision up to $0.01mm$.
The authors examined the sensor's data and possible usage as the gesture interface utilizing two types of action gestures: static and dynamic.
The static gestures understood as poses of a hand and fingers are recognized using the Support Vector Machine (SVM) with intelligent pre- and postprocessing.
The proposed approach allowed to recognize five gestures with $99\%$ accuracy and ten gestures with $85\%$.
The dynamic gestures that are movements of a hand and fingers in time are recognized with the Hidden Markov Models (HMM). 
The adopted approach allowed to achieve accuracy up to $80\%$ for five dynamic gestures.
The thesis contains also experiments of the proposed finger recognition module working with $93\%$ accuracy. This module is also based on SVM.
The main outcome of the thesis is the open-source library LeapGesture dedicated to the developers for Leap Motion Controller, that supports parameterized gestures, apart from the recognition of standard action gestures and contains presented approaches using a high-level C++ interface making gesture recognition easy to utilize in any application.