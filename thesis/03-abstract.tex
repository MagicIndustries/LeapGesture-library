\chapter{Streszczenie}
Niniejsza praca jest poświęcona nowym możliwościom zastosowania gestów do interakcji człowieka z komputerem, które pojawiły się wraz z wprowadzeniem na rynek urządzenia Leap Motion.
Leap Motion to innowacyjne urządzenie, które udostępnia dane dotyczące pozycji rąk oraz palców w przestrzeni trójwymiarowej z dokładnością do $0,01$mm.
Głównym rezultatem niniejszej pracy jest przeznaczona dla programistów, dedykowana dla kontrolera Leap Motion, biblioteka LeapGesture udostępniona na licencji open-source, która zawiera algorytmy do nauki oraz rozpoznawania gestów.
Autorzy pracy zbadali możliwości sensora w zastosowaniu rozpoznawania układów dłoni (gestów statycznych), wykonywanych ruchów ręki (gestów dynamicznych) oraz rozróżniania palców.
Statyczne gesty są rozpoznawane za pomocą mechanizmu maszyn wektorów wspierających (SVM) z wykorzystaniem filtracji medianowej danych wejściowych oraz przy wykorzystaniu zależności pomiędzy kolejnymi rozpoznanymi gestami w danym oknie czasowym.
Praca zawiera także badania różnych używanych wektorów cech, których wybór ma znaczny wpływ na uzyskiwane rezultaty.
Zaproponowane podejście do wybranych cech umożliwiło rozpoznawanie zbioru gestów należących do pięciu klas z dokładnością $99\%$ oraz zbioru gestów należących do dziesięciu klas z dokładnością $85\%$.
Gesty dynamiczne (ruch ręki oraz palców) są rozpoznawane za pomocą ukrytych modeli Markowa (HMM).
Przyjęte podejście umożliwiło osiągnięcie $80\%$ skuteczności rozpoznawania gestów dynamicznych należących do sześciu klas.
Moduł rozróżniania palców ręki dla badanych zbiorów osiągnął dokładność rozpoznawania wynoszącą $93\%$. 
W bibliotece zostały zaimplementowane wyżej wymienione podejścia w języku C++, dzięki czemu biblioteka LeapGesture może być użyta w aplikacji każdego typu.

\chapter{Abstract}
Since the invention of computers there exists a need to develop more intuitive human-computer interfaces.
There has been keyboards, mouses, but those solutions are not as natural as gestures, which are essential part of a human expression.
This thesis studies the new possibilities to gesture interfaces that emerged with a Leap Motion sensor.
The work also proposes a new gesture classification into action and parameterized gestures, which is defined strictly in the context of gesture recognition.
The Leap Motion is an innovative, 3D motion capturing device designed especially for hands and fingers tracking with precision up to $0.01$mm.
The authors examined the sensor's data and possible usage as the gesture interface utilizing two types of action gestures: static and dynamic.
The static gestures understood as poses of a hand and fingers are recognized using the Support Vector Machine (SVM) with intelligent pre- and postprocessing.
The proposed approach allowed to recognize five gestures with $99\%$ accuracy and ten gestures with $85\%$.
The dynamic gestures that are movements of a hand and fingers in time are recognized with the Hidden Markov Models (HMM). 
The adopted approach allowed to achieve accuracy up to $80\%$ for five dynamic gestures.
The thesis contains also experiments of the proposed finger recognition module working with $93\%$ accuracy. This module is also based on SVM.
The main outcome of the thesis is the open-source library LeapGesture dedicated to the developers for Leap Motion Controller, that supports parameterized gestures, apart from the recognition of standard action gestures and contains presented approaches using a high-level C++ interface making gesture recognition easy to utilize in any application.