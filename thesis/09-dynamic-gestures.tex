\chapter{Detection of dynamic gestures}

\section{Proposed methods}

The dynamic gesture recognition problem is a problem, where the input data consist of several consecutive positions and orientations of hand and fingers. 
Moreover, the important factor for recognition is the time dependencies between data frames. The slower and faster gestures should be recognized as the same dynamic gesture.

The proposed solution utilizes parts of the solution used for recognition of the static gestures.
Each frame of the captured data is described by the same features as in the static recognition part.
The set of features for each frame is then processed by the Hidden Markov Model scheme. 

\subsubsection{Hidden Markov Model}

Hidde Markov Models can be represented by the set:
$ ( , , , , , )$

where $O$ is set of observations, $X$ is a set of states, $Y$ is a matrix defining the probabilities of transitions between states, $Z$ is a matrix defining the probabilities of observations from each state.
The best way to represent HMM is to use the structure of the graph with two types of vertices. 
This way, each state is represented by one type of vertices while observations can be shown as second type of vertices.
The edges between states contain and are an equivalent to the e$Y$ matrix. 
There are no edges between vertices representing observations.
The edges between states and observations also contain probabilities from the $Z$ matrix.
The problem of recognizing the dynamic gesture can be understood as a problem of finding the path of states in HMM which maximizes the combined probability given set of observations. It can be written as:
$ xx $
There are three main algorithms connected with HMM:
\begin{itemize}
\item Forward-Backward algorithm, 
\item Viterbi algorithm,
\item Baum-Welch algorithm.
\end{itemize}

The Forward-Backward algorithm is used to find the state that maximizes the likelihood given the set of observations.
It achieves that by performing two operations. First of them **** .

The Viterbi algorithm is used the path of connected states that best explain the set of observations. 
It is algorithm based on the Forward-Backward, which also can understood as the usage of the Expected Maximization approach.

The Baum-Welch algorithm is the algorithm used to train the HMM.
For each training example, the algorithms changes the probability and transition matrices to maximize the likelihood of observation.

In dynamic gesture recognition problem, each gesture can be modelled by the sequence of $n$ states in which $kths$ state is connected by the edges to the $k$ and $(k+1)$ state and to the all observations.
The proposed architecture can be seen at fig. *. 
Having the problem of distinguishing $m$ gestures translates to the $m$ sequential graphs.







\section{Evaluation methodology}







  






The proposed solution utilizes Hidden Markov Model[] 



\section{Experiments}
