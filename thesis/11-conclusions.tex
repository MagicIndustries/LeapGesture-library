\chapter{Conclusions}\label{conclusionsChapter}

This thesis describes LeapGesture, which is the hand gesture recognition library dedicated for Leap Motion Controller.
This library includes modules for static and dynamic gesture recognition and also finger recognition.
Developed library provides recognition of all types of action gestures.
Static action gestures are supported by static gesture processing thread, while the dynamic gesture processing thread provides recognition of dynamic action gestures.
Additionally, comparison of existing gesture recognition methods, implementation of additional modules enabling the recording and reviewing of gestures, creation of sample gestures database and performation of tests has been conducted.
%This thesis contains a thorough evaluation of proposed approaches to the gesture recognition.

The results obtained for the static gesture recognition suggest than those gestures are easy to recognize using the SVM with appropriate pre and postprocessing.
The differences between the recognition rates for different feature sets even values reaching up to $13\%$ show that choosing different feature sets can have a major impact on the performance of recognition.
Additionally, it is important not to undermine the influence of the data preprocessing. 
Even the usage of the relatively simple median filter allowed to boost the recognition rate by $6\%$.
Although, the authors believe that the proposed parameters of processing module work well in many scenarios, there might be some specific cases that need different parameters or different feature sets to achieve good results.
It is also worth noting, that this and any another approach will not work properly if during the performance of static gestures, hands and fingers are not correctly detected by the Leap Motion.
Therefore it is recommended to always choose gestures that have visible and separated fingers from the sensor's perspective.
For the proposed approach, recognition rate of $99\%$ for five classes of gestures and $85\%$ for ten classes of gestures in the static gesture recognition task were achieved. Satisfactory results of $93\%$ were also obtained for the task of finger recognition for $15$ of $32$ classes of fingers arrangements.

When it comes to the task of dynamic gestures recognition, Leap Motion might not be the best sensor choice.
Leap Motion provides great accuracy for static hand and fingers, but the data for dynamically moved hands are usually noisy.
The problems arise with short, lost finger tracking or temporal finger occlusions that are hard to detect and cope with on the library-level.
The proposed preprocessing module tries to alleviate those negative impacts, but the preprocessed data is still far from ideal.
Nevertheless the proposed approach with Hidden Markov Models allowed to recognize five classes of dynamic gestures with $80\%$ accuracy.

As part of the further development of the library, the authors intend to improve the efficiency of dynamic gesture recognition by testing other possible feature sets and to develop methods that will allow to preprocessed data derived from Leap Motion more efficiently. Supporting of parameterized gestures is also envisaged.

